\documentclass[a4paper,11pt]{article}
\usepackage[utf8]{inputenc}
\usepackage[margin=1in]{geometry}
\usepackage{multicol}
\usepackage{enumitem}
\usepackage{parskip}
\usepackage{hyperref}
\usepackage{array}

\setlist[itemize]{noitemsep, topsep=0pt, leftmargin=*}
\setlength{\columnsep}{0.5cm}

\hypersetup{colorlinks=true, urlcolor=black}

\begin{document}

\vspace*{-6.5em} 
\begin{center}
    \href{https://menasy.com.tr/}{\LARGE \textbf{Mehmet Nasim Yılmaz}}
\end{center}
\vspace*{0.5em} 
\noindent
\begin{minipage}[t]{0.48\textwidth}
    \begin{tabular}{@{}l}
        \textbf{Lokasyon:} İstanbul \\
        \textbf{Doğum Tarihi:} 02.02.2000 \\
        \textbf{Github:} \href{https://github.com/menasy}{github.com/menasy} \\
        \textbf{Portfolio:} \href{https://menasy.com.tr/}{menasy.com.tr} \\
    \end{tabular}
\end{minipage}%
\hfill
\begin{minipage}[t]{0.48\textwidth}
    \raggedleft
    \begin{tabular}{@{}l}
        \textbf{Numara:} +90 5441734873 \\
        \textbf{E-posta:} mehmetsin42@gmail.com \\
        \textbf{Linkedin:} \href{https://linkedin.com/in/menasy}{linkedin.com/in/menasy} \\
    \end{tabular}
\end{minipage}

\vspace{0.7em}

% Özet
Yazılım geliştirme, web teknolojileri, DevOps ve özellikle mobil programlama alanlarında kendimi sürekli geliştirmeye odaklanıyorum. Bu süreçte; C, C++ ve Java ile çeşitli projeler geliştirdim. Android tabanlı uygulamalar yaptım. Docker ve konteyner teknolojilerini kullanarak mikro servis mimarileri kurdum. C++ ile soket programlama yaparak, ekip arkadaşım ile HTTP/1.1 protokolünü kullanan bir web sunucusu geliştirdik. Tüm bu projelerle ilgili detaylara, \textbf{\href{https://medium.com/@menasy}{Medium}} yazılarıma ve diğer çalışmalarımı içeren \textbf{\href{https://menasy.com.tr/}{portföyüme}} göz atabilirsiniz.
\vspace{-0.5em}

Bugüne kadar edindiğim teknik ve pratik bilgiyle sağlam bir altyapı oluşturduğuma inanıyorum. Şimdi ise bu bilgi ve tecrübeyi iş hayatına taşıyarak hem çalıştığım şirkete katkı sağlamak hem de kendimi daha ileri bir seviyeye taşımayı hedefliyorum.
\vspace{-0.5em}

\textit{"Bilmediğim hiçbir şey yok, sadece öğrenmediğim şeyler var."} anlayışıyla kendimi sürekli geliştirmeye devam ediyorum ve asla pes etmiyorum. Şirketinizde çalışmak benim için büyük bir fırsat olacaktır. İlginiz için teşekkür eder, saygılarımı sunarım.

\vspace{0.4em}

\textbf{Eğitim Bilgileri}
\vspace{-0.2em}
\begin{itemize}[leftmargin=2em]
  \item 42 İstanbul – Bilişim Teknolojileri Mimarisi Uzmanı / 2023 - 2025
  \item İstanbul Gelişim Üniversitesi – Yönetim Bilişim Sistemleri / 2021 - 2025
\end{itemize}

\vspace{-0.1em}

\textbf{Deneyimler}
\vspace{-0.2em}
\begin{itemize}[leftmargin=2em]
  \item T3 Vakfı – Mobil Uygulama Geliştirme Contribütörü (Next Sosyal) / 01.07.2025 -
  \item Paka Teknoloji – Sistem Destek Stajyeri / 12.08.2024 – 23.09.2024
  \item Tencent Holdings Limited – Sunucu Teknisyeni / 01.04.2023 – 01.05.2023
\end{itemize}

\vspace{0.1em}

% İlk satır: Programlama Dilleri & Programlama Becerileri
\noindent
\begin{minipage}[t]{0.48\textwidth}
    \textbf{Programlama Dilleri}
    \vspace{0.5em}
    \begin{itemize}[leftmargin=2em]
        \item C (expert)
        \item C++ (expert)
        \item Java (middle)
        \item Python (middle)
        \item Bash Script (middle)
        \item XML (middle)
        \item Kotlin (begin)
        \item C\# (begin)
        \item SQL (begin)
		\item JavaScript/React (begin)
		\item HTML/CSS (begin)
        \item Solidity (begin)
    \end{itemize}
\end{minipage}%
\hfill
\begin{minipage}[t]{0.48\textwidth}
    \textbf{Programlama Becerileri}
    \vspace{0.5em}
    \begin{itemize}[leftmargin=2em]
        \item Multi Process ve Multi Thread
        \item Nesne Yönelimli Programlama
        \item Mobil Uygulama Geliştirme
        \item Docker ve Docker Compose
		\item Sistem Programlama
        \item Socket Programlama
		\item Web Geliştirme
		\item Smart Contract
        \item Git/GitHub	
		\item REST API
    \end{itemize}
\end{minipage}


\vspace{0.3em}
\noindent
\begin{minipage}[t]{0.48\textwidth}
    \textbf{İlgi Alanları}
    \vspace{0.5em}
    \begin{itemize}[leftmargin=2em]
        \item Mobil Uygulama Geliştirme
		\item Web Teknolojileri
        \item DevOps
    \end{itemize}
\end{minipage}%
\hfill
\begin{minipage}[t]{0.48\textwidth}
    \textbf{Kodlama Araçları}
    \vspace{0.5em}
    \begin{itemize}[leftmargin=2em]
        \item Visual Studio Code
        \item Jupyter Notebook
        \item Android Studio
        \item Remix IDE
        \item Pycharm
        \item Vim
    \end{itemize}
\end{minipage}

% Üçüncü satır: Platformlar & Diller

\vspace{0.3em}
\noindent
\begin{minipage}[t]{0.48\textwidth}
    \textbf{Platformlar}
    \vspace{0.5em}
    \begin{itemize}[leftmargin=2em]
        \item Linux
        \item Numpy
        \item Pandas
        \item Android
    \end{itemize}
\end{minipage}%
\hfill
\begin{minipage}[t]{0.48\textwidth}
    \textbf{Diller}
    \vspace{0.5em}
    \begin{itemize}[leftmargin=2em]
        \item İngilizce (B1)
    \end{itemize}
\end{minipage}

\end{document}