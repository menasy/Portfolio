\documentclass[a4paper,11pt]{article}
\usepackage[utf8]{inputenc}
\usepackage[T1]{fontenc}
\usepackage[turkish]{babel}
\usepackage[margin=1in]{geometry}
\usepackage{multicol}
\usepackage{enumitem}
\usepackage{parskip}
\usepackage{hyperref}
\usepackage{array}

\setlist[itemize]{noitemsep, topsep=0pt, leftmargin=2em}
\setlength{\columnsep}{0.5cm}

\hypersetup{colorlinks=true, urlcolor=black}

\begin{document}

\vspace*{-6.5em}
\begin{center}
    \href{https://menasy.com.tr/}{\LARGE \textbf{Mehmet Nasim Yılmaz}}
\end{center}
\vspace*{0.5em}
\noindent
\begin{minipage}[t]{0.48\textwidth}
    \begin{tabular}{@{}l}
        \textbf{Lokasyon:} İstanbul \\
        \textbf{Github:} \href{https://github.com/menasy}{github.com/menasy} \\
        \textbf{Portfolio:} \href{https://menasy.com.tr/}{menasy.com.tr} \\
    \end{tabular}
\end{minipage}%
\hfill
\begin{minipage}[t]{0.48\textwidth}
    \raggedleft
    \begin{tabular}{@{}l}
        \textbf{Doğum Tarihi:} 02.02.2000 \\
        \textbf{E-posta:} mehmetsin42@gmail.com \\
        \textbf{Linkedin:} \href{https://linkedin.com/in/menasy}{linkedin.com/in/menasy} \\
    \end{tabular}
\end{minipage}

\vspace{0.5em}

% Özet
Yazılım geliştirme; web teknolojileri, DevOps ve mobil programlama alanlarında kendimi sürekli geliştirmeye odaklanıyorum. Bu süreçte; android tabanlı uygulamalar yaptım. Docker kullanarak mikro servis mimarileri kurdum. Baykar'daki stajım boyunca Wowza Streaming Engine, .NET tabanlı backend ve React tabanlı frontend kullanarak full-stack bir canlı yayın platformu geliştirdim. C++ ile soket programlama yaparak, ekip arkadaşımla sıfırdan; HTTP/1.1 protokolünü kullanan bir web sunucusu yaptık. Klasik Pong oyununu frontend kısmını ben üstlenerek ekip arkadaşlarımla birlikte modern teknolojilerle sıfırdan geliştirdik. Tüm bu projelerle ilgili detaylara, \textbf{\href{https://medium.com/@menasy}{Medium}} yazılarıma ve diğer çalışmalarımı içeren \textbf{\href{https://menasy.com.tr/}{portföyüme}} göz atabilirsiniz.
\vspace{-0.3em}

Bugüne kadar edindiğim teknik ve pratik bilgiyle sağlam bir altyapı oluşturduğuma inanıyorum. Şimdi ise bu bilgi ve tecrübeyi iş hayatına taşıyarak hem çalışacağım şirkete katkı sağlamak hem de kendimi daha ileri bir seviyeye taşımayı hedefliyorum.
\vspace{-0.3em}

\textit{"Bilmediğim hiçbir şey yok, sadece öğrenmediğim şeyler var."} anlayışıyla kendimi sürekli geliştirmeye devam ediyor ve asla pes etmiyorum. Şirketinizde çalışmak benim için büyük bir fırsat olacaktır. İlginiz için teşekkür eder, saygılarımı sunarım.

\vspace{-0.1em}

\textbf{Eğitim Bilgileri}
\vspace{-0.2em}
\begin{itemize}
  \item 42 İstanbul – Yazılım Mühendisliği / 2023 - 2025
  \item İstanbul Gelişim Üniversitesi – Yönetim Bilişim Sistemleri / 2021 - 2025
\end{itemize}

\vspace{-0.1em}

\textbf{Deneyimler}
\vspace{-0.2em}
\begin{itemize}
  \item Baykar Teknoloji – Merkezi Kontrol Yazılımları Stajyeri / 06.10.2025 - 27.12.2025
  \item Paka Teknoloji – Sistem Destek Stajyeri / 12.08.2024 – 23.09.2024
  \item Tencent Holdings Limited – Sunucu Teknisyeni / 01.04.2023 – 01.05.2023
\end{itemize}

\vspace{0.1em}

% İlk satır: Programlama Dilleri & Programlama Becerileri
\noindent
\begin{minipage}[t]{0.48\textwidth}
    \textbf{Programlama Dilleri}
    \vspace{0em}
    \begin{itemize}
        \item C (expert)
        \item C\#/ASP.NET (middle)
        \item C++ (middle)
        \item Java (middle)
        \item JavaScript/TypeScript/React (middle)
        \item HTML/Tailwind/CSS (middle)
        \item SQL/PostgreSQL (middle)
        \item Python (middle)
        \item XML (middle)
        \item Bash Script (begin)
        \item Kotlin (begin)
        \item Solidity (begin)
    \end{itemize}
\end{minipage}%
\hfill
\begin{minipage}[t]{0.48\textwidth}
    \textbf{Programlama Becerileri}
    \vspace{0em}
    \begin{itemize}
        \item Multi Process ve Multi Thread
        \item Nesne Yönelimli Programlama
        \item Mobil Uygulama Geliştirme
        \item Mikro Servis Mimarisi
        \item Full-Stack Geliştirme
        \item Sistem Programlama
        \item Socket Programlama
        \item Docker Teknolojisi
        \item Medya Sunucusu
        \item Smart Contract
        \item MVC Mimarisi
        \item Git/GitHub
        \item REST API
    \end{itemize}
\end{minipage}

\vspace{0.1em}
\noindent
\begin{minipage}[t]{0.48\textwidth}
    \textbf{İlgi Alanları}
    \vspace{0em}
    \begin{itemize}
        \item Mobil Uygulama Geliştirme
        \item Full-Stack Geliştirme
        \item Web Teknolojileri
        \item DevOps
    \end{itemize}
\end{minipage}%
\hfill
\begin{minipage}[t]{0.48\textwidth}
    \textbf{Kodlama Araçları}
    \vspace{0em}
    \begin{itemize}
        \item Visual Studio Code
        \item Jupyter Notebook
        \item Android Studio
        \item Remix IDE
        \item Pycharm
        \item Vim
    \end{itemize}
\end{minipage}

% Üçüncü satır: Platformlar & Diller
\vspace{0.2em}
\noindent
\begin{minipage}[t]{0.48\textwidth}
    \textbf{Platformlar}
    \vspace{0em}
    \begin{itemize}
        \item Linux
        \item Pandas
        \item Wowza
        \item Android
    \end{itemize}
\end{minipage}%
\hfill
\begin{minipage}[t]{0.48\textwidth}
    \textbf{Diller}
    \vspace{0em}
    \begin{itemize}
        \item İngilizce (B1)
    \end{itemize}
\end{minipage}

\end{document}
